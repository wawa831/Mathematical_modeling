\documentclass{ctexart}
\usepackage{graphicx}
\usepackage{amsmath}
\usepackage{listings}
\usepackage{caption}
\title{基于交通网络的城市规划项目评估与推荐}
\author{ICM团队编号}
\date{\today}

\begin{document}

\maketitle

 **摘要**
随着城市发展,交通系统对城市及利益相关者的影响显著增加。本文构建交通网络模型,评估不同项目对利益相关者的影响,并推荐提升居民生活质量的方案。首先建立行人和驾驶员交通网络,通过整合地铁因素优化行人网络,基于交通流量定义道路拥堵程度并计算边权。针对桥梁坍塌、新增公交站和新建地铁线等场景,分别采用非均匀连续时间随机游走、中心性分析和模糊综合评价等方法,量化项目对居民、企业和政府的影响,最终通过敏感性分析验证模型鲁棒性。

**关键词**:交通网络;随机游走;模糊评价;层次分析法


 **1. 问题背景**
高效的交通系统是城市发展的核心。以巴尔的摩为例,其老旧的交通基础设施和桥梁坍塌事件加剧了交通拥堵,制约经济增长并影响居民生活。通过优化交通网络提升可达性、降低通勤成本,是实现城市可持续发展的关键目标。


 **2. 模型准备**
 **2.1 假设条件**
1. 街道通行时间仅与道路属性相关,不受环境因素影响。
2. 行人和驾驶员均按固定速度行驶,不随意停留或折返。
3. 行人和车辆网络相互独立,无交叉路径。

 **2.2 数据预处理**
- **数据筛选**:根据道路类型和通行权限过滤无效数据,如图1所示。
- **缺失值填充**:采用反距离插值法估算缺失的交通流量数据,公式为:
  \[
  \hat{T}_{ij} = \frac{\sum_{k=1}^{n} \frac{T_k}{d_k}}{\sum_{k=1}^{n} \frac{1}{d_k}}
  \]
  其中,\(d_k\)为待填充街道与已知数据街道的距离,\(T_k\)为已知交通流量。

 **3. 交通网络构建**
**3.1 行人交通网络**
- **无权网络**:边权定义为步行时间,\(W_{ij}^P = \frac{L_{ij}}{V^W}\),其中\(V^W = 1.5 \, \text{m/s}\)为步行速度。
- **地铁整合**:引入地铁线路,边权包含行驶时间和固定等待时间,\(W_{ij}^S = \frac{L_{ij}}{V^S} + 15 \, \text{min}\)。

**3.2 驾驶员交通网络**
- **有向网络**:边权结合车道数、交通流量和拥堵系数,采用Sigmoid函数映射拥堵程度:
  \[
  W_{ij}^D = \frac{L_{ij} \cdot N_{ij} \cdot \epsilon}{V^D}, \quad \epsilon = 1 + \frac{1}{1 + e^{-0.00005(D_{ij} - 12643)}}
  \]
  其中,\(D_{ij}\)为日均交通量,\(\epsilon\)随拥堵程度递增。


**4. 桥梁坍塌影响评估**
采用非均匀连续时间随机游走模型,结合最短路径概率(80%选择最短路径,20%随机选择其他路径),计算坍塌前后各节点到达四大繁忙区域的平均时间。结果表明,驾驶员网络受影响较小,而桥梁周边局部网络的行人通勤时间显著增加,反映对居民跨区域出行的负面影响。

 **5. 新增公交站项目分析**
通过度中心性确定居民中心,利用介数中心性和紧密中心性筛选关键公交站。在居民密集区新增10个站点后,居民到最近公交站的平均距离从10587米缩短至5070米,平均到达时间减少52%,表明公交覆盖和可达性显著提升,尤其惠及郊区居民。


 **6. 新建地铁线推荐**
构建连接机场和港口的地铁线路,采用层次分析法(AHP)确定指标权重,通过模糊综合评价量化改善程度。结果显示,项目对居民便利性(得分80.86)和企业外贸(得分86.12)提升显著,但政府管理成本指标得分较低(37.29),需关注施工期间的短期影响。


 **7. 敏感性分析**
调整居民相关指标权重(便利性、环境、安全),发现总得分波动在可控范围内(误差<5%),验证模型鲁棒性。结果表明,权重差异对评估等级无显著影响,模型具有可靠性。


 **8. 模型优缺点**
- **优点**:结合多模式交通网络,数据处理科学,适用于复杂城市场景。
- **缺点**:未考虑社区聚集效应,随机游走模型对网络连通性要求较高。


 **9. 致市长备忘录**
推荐新增公交站和地铁线项目,指出前者提升本地通勤便利性,后者促进外贸经济。建议优先实施地铁项目,同时优化施工方案以减少短期拥堵,长期可提升城市竞争力和居民满意度。

\end{document}
